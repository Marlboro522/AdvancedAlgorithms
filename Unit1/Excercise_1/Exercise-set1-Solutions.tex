\documentclass[12pt]{article}
\usepackage{amsmath}
\usepackage{geometry}[margin=2cm]
\begin{document}
\title{Exercise Set 1: Big O Notation}
\author{Raja Kantheti}
\date{}
\maketitle

\section{Question 1: }
\subsection*{ f(n) = $3n^2 + 7n + 10$}
Big O classiffication: O($n^2$)
\subsection*{f(n)=$5n\log(n)+n^2$}
Big O classification: O($n^2$)
\subsection*{f(n)=$2^n + n^2$}
Big O classification: O($2^n$)

\section{Question 2: }
Growth Rates from slowest to fastest:
$n\log(n) \longrightarrow n^2 \longrightarrow 2^n \longrightarrow n!$

\section{Question 3: }

We define the function:

\[
f(n) = n^3 + 100n^2 + 50n + 10
\]

To prove \( f(n) = O(n^3) \), we need to find constants \( c > 0 \) and \( n_0 \) such that:

\[
f(n) \leq c n^3, \quad \forall n \geq n_0.
\]

Replace each term with an upper bound proportional to \( n^3 \):

\[
n^3 + 100n^2 + 50n + 10 \leq n^3 + 100n^3 + 50n^3 + 10n^3.
\]

Since for \( n \geq 1 \), we can deduct that, \( n^2 \leq n^3 \), \( n \leq n^3 \), and \( 1 \leq n^3 \)

\[
= (1 + 100 + 50 + 10)n^3 = 161n^3.
\]

Thus, we have:

\[
f(n) \leq 161 n^3, \quad \forall n \geq 1.
\]

By setting \( c = 161 \) and \( n_0 = 1 \), we have a c and an $n_0$ which satisfy the definition of big O notation. Therefore, we can conclude that:

\[
f(n) = O(n^3).
\]
\section{Question 4: }
\subsection*{For SElection sort: }
runns inner and ourter oop irrespective of the input.(Worst and best input)\\
Worst-case scenario: $n^2$\\
Best-case scenario: $n^2$


\subsection*{For Insertion sort: }
Runs only the outer loop for input that is sorted.(Best Input)\\
RUns both inner and outer loop for input that is reverse sorted or random.(Worst input)\\
Worst-case scenario: $n^2$\\
Best-case scenario: $n$
\end{document}